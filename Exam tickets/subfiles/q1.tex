\documentclass[../document.tex]{subfiles}

\begin{document}
\section{Модель множественной регрессии. Основные предположения регрессионной модели. Оценка коэффициентов методом наименьших квадратов}
	\subsection{Модель множественной линейной регресси}
	\par Модель множественной линейной регрессии предназначена для проверки и изучения связи между одной зависимой переменной и несколькими независимыми  переменными. Предполагается, что такая связь теоретически может быть описана  функцией вида: $Y = \beta_1x_1 + \beta_2x_2 + ... + \beta_kx_k + U$
	
	\subsection{Основные предположения регрессионной модели}
	\begin{enumerate}
		\item $y = X\beta + \epsilon$ - линейная спецификация модели
		\item $X$ - детерменированная матрица максмального ранга $k$
		\item \begin{enumerate}
			\item $E(\mathcal{E}) = 0, V(\mathcal{E}) = E(\mathcal{E}^T\mathcal{E}) = \sigma^2I_n$
			\item $\mathcal{E} \sim N(0, \sigma^2I_n)$
		\end{enumerate}
	\end{enumerate}
\end{document}